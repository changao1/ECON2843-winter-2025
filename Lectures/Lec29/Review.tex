\documentclass[12pt]{beamer}
\usetheme{Boadilla}
\usepackage{booktabs}
\usepackage{multirow}
\usepackage{enumitem}
\usepackage{tikz}
\usepackage[normalem]{ulem} 
\newcommand{\E}{\mathbb{E}}
\usefonttheme{professionalfonts}
\usepackage{pgfplots}
\pgfplotsset{compat=1.18}
\renewcommand{\arraystretch}{1.25}
\usetikzlibrary{trees}
\title[ECON2843]{Lecture 29}
\subtitle{Review Session}
\date{}
\usepackage{amsmath,amssymb,mathtools,wasysym}
\begin{document}
	\begin{frame}
		\titlepage
		
	\end{frame}
	\begin{frame}
		\frametitle{Here is the Plan for Final Weeks}
		\begin{itemize}[label={\color{blue}$\blacktriangleright$}]
			\item This week
			\begin{itemize}[label={\color{blue}$\blacktriangleright$}]
			\item Tuesday: Last class, review session and some announcement
			\item Thursday: No class, Q \& A session in Room 236, Cate Center 1 during regular class time.
			\end{itemize}
			\item Next Week
			\begin{itemize}[label={\color{blue}$\blacktriangleright$}]
				\item Tuesday: No class, Q \& A session in Room 236, Cate Center 1 from 1:30pm - 2:30pm.
				\item Wednesday 12:30pm - 6:30pm, window for the optional final exam.
			\end{itemize}
				\end{itemize}

	\end{frame}
	
		\begin{frame}
		\frametitle{Clarification on Optional Final Exam}
{\large\color{red} It is truly optional!}
			\begin{itemize}[label={\color{blue}$\blacktriangleright$}]
				\item Calculate your grades in the following manner:
				\begin{itemize}[label={\color{blue}$\blacktriangleright$}]
					\item Take the arithmetic mean of your highest 15 out of the 18 assignments.
					\item For assignment with technical issues, make sure confirm with me through email about excluding it.
					\item For assignments with technical issues, you can calculate the grade by taking the arithmetic mean of your highest $15-x$ out of the $18-x$ assignments, where $x$ is number of  assignments with issues (that you have confirmed with me).
					
				\end{itemize}

				
					\end{itemize}
			
			\end{frame}
					\begin{frame}
					\frametitle{Clarification on Optional Final Exam}
			Take the optional final if
			\begin{itemize}[label={\color{blue}$\blacktriangleright$}]
				\item You are not happy with the current cumulative grade.
				\item You are interested in taking it.
				\item In the end, I will take the higher one between your assignments average and optional final.
			\end{itemize}

	\end{frame}
	\begin{frame}
		\frametitle{About the Optional Final Exam Format}
		\begin{itemize}[label={\color{blue}$\blacktriangleright$}]
			\item It will be given between 12:30pm - 6:30pm on December 11th, which is next Wednesday.
			\item It has 30-35 T/F and multiple choices questions.
			\item You have 100 minutes to complete it.
			\item Covers material until multiple linear regression.
			\item You can start at any time between 12:30pm - 4:50pm, after 4:50pm might be too late.
			\item Open book exam, it is similar to the quizzes except the length.
		\end{itemize}
		
	\end{frame}
	\begin{frame}
	\frametitle{Curve and Attendance Policy}
	Curving
	\begin{itemize}[label={\color{blue}$\blacktriangleright$}]
		\item As mentioned in syllabus, I will curve the median grade to 90 if it is below 90.
		\item I will do the curving by moving the bell shape grade distribution toward right.
	\end{itemize}
	Attendance
	\begin{itemize}[label={\color{blue}$\blacktriangleright$}]
	\item I won't deduct your marks for missing classes.
\item I will add marks in calculating letter grades for those show up in classes.
	\end{itemize}

\end{frame}
	\begin{frame}
	\frametitle{The Course Reflection Survey}
	\begin{itemize}[label={\color{blue}$\blacktriangleright$}]
\item I'd like to ask you to do me a favor. If possible, could you please help me with the Course Reflection Survey? It is due by this Sunday.

\item Then I will know where I should improve.

\item Thank you!
	\end{itemize}
\end{frame}
\begin{frame}
	\frametitle{Data Types and Statistics}
	\[
	\begin{array}{l}
		\text{Data} 
		\left\{
		\begin{array}{l}
			\text{Categorical}
			\left\{
			\begin{array}{l}
				\text{nominal} \\[1ex]
				\text{ordinal}
			\end{array}
			\right. \\[2ex]
			\text{Numerical}
			\left\{
			\begin{array}{l}
				\text{discrete} \\[1ex]
				\text{continuous}
			\end{array}
			\right.
		\end{array}
		\right. \\[4ex]
		\text{Population - parameters}\ \text{\scriptsize(unknown)}
		\left\{
		\begin{array}{l}
			\text{mean} \\
			\text{variance}
		\end{array}
		\right. \\[3ex]
		\text{Sample - statistics}\ \text{\scriptsize(calculated)}
		\left\{
		\begin{array}{l}
			\text{mean} \\
			\text{variance}
		\end{array}
		\right.
	\end{array}
	\]
\end{frame}
\begin{frame}
	\frametitle{Hypothesis Testing}
	
	\begin{itemize}[label={\color{blue}$\blacktriangleright$}]
		\item Define your null ($H_0$) and alternative ($H_1$) hypotheses.
		
		\item Calculate an appropriate test statistic.
		
		\item Based on the sampling distribution of the test statistic under $H_0$, reject $H_0$ if the observed test statistic is extreme (using rejection regions or $p$-values).
		
		\item Reject $H_0$ if your test statistic falls in the rejection region or if your $p$-value is less than $\alpha$.
	\end{itemize}
	
\end{frame}
\begin{frame}
	\frametitle{Single Population}
	
	\begin{itemize}[label={\color{blue}$\blacktriangleright$}]
		\item Testing $\mu$ when $\sigma^2$ known:
		\begin{itemize}[label={\color{blue}$\blacktriangleright$}]
			\item Calculate $Z$-statistic.
			\item Compare to $N(0,1)$ distribution.
			\item One or two-tailed depending on $H_1$.
		\end{itemize}
		
		\item Testing $\mu$ when $\sigma^2$ unknown:
		\begin{itemize}[label={\color{blue}$\blacktriangleright$}]
			\item Calculate $T$-statistic.
			\item Compare to $t$-distribution with $n-1$ degrees of freedom.
			\item One or two-tailed depending on $H_1$.
		\end{itemize}
		
		\item Testing a population proportion $p$:
		\begin{itemize}[label={\color{blue}$\blacktriangleright$}]
			\item Calculate $Z$-statistic.
			\item Compare to $N(0,1)$ distribution.
			\item One or two-tailed depending on $H_1$.
		\end{itemize}
	\end{itemize}
	
\end{frame}
\begin{frame}
	\frametitle{Comparing Two Populations}
	\framesubtitle{Independent Samples}
	
	\begin{itemize}[label={\color{blue}$\blacktriangleright$}]
		\item Testing $\mu_1 - \mu_2$ when $\sigma_1^2$, $\sigma_2^2$ known:
		\begin{itemize}[label={\color{blue}$\blacktriangleright$}]
			\item Calculate $Z$-statistic.
			\item Compare to $N(0,1)$ distribution.
			\item One or two-tailed depending on $H_1$.
		\end{itemize}
		
		\item Testing $\mu_1 - \mu_2$ when $\sigma_1^2$, $\sigma_2^2$ unknown and $\sigma_1^2 = \sigma_2^2$:
		\begin{itemize}[label={\color{blue}$\blacktriangleright$}]
			\item Calculate pooled sample variance $s_p^2$.
			\item Calculate $T$-statistic.
			\item Compare to $t$-distribution with $n_1 + n_2 - 2$ degrees of freedom.
			\item One or two-tailed depending on $H_1$.
		\end{itemize}
	\end{itemize}
	
\end{frame}
\begin{frame}
	\frametitle{Comparing Two Populations}
	\framesubtitle{Independent Samples}
	
	\begin{itemize}[label={\color{blue}$\blacktriangleright$}]
		\item Testing $H_0 : \sigma_1^2 = \sigma_2^2$ vs $H_1 : \sigma_1^2 \neq \sigma_2^2$:
		\begin{itemize}[label={\color{blue}$\blacktriangleright$}]
			\item Calculate $F$-statistic (put larger sample variance on top).
			\item Compare to $F$-distribution with $n_1 - 1$ numerator degrees of freedom and $n_2 - 1$ denominator degrees of freedom ($n_1$ is the sample size corresponding to the larger sample variance).
			\item Two-tailed, but only need to look at upper tail of $F$-distribution.
		\end{itemize}
	\end{itemize}
	
\end{frame}
\begin{frame}
	\frametitle{Comparing Two Populations}
	\framesubtitle{Paired Samples}
	
	\begin{itemize}[label={\color{blue}$\blacktriangleright$}]
		\item Testing $\mu_D$:
		\begin{itemize}[label={\color{blue}$\blacktriangleright$}]
			\item Calculate paired differences (remember to set up hypotheses appropriately).
			\item Calculate $T$-statistic.
			\item Compare to $t$-distribution with $n - 1$ degrees of freedom.
			\item One or two-tailed depending on $H_1$.
		\end{itemize}
	\end{itemize}
	
\end{frame}
\begin{frame}
	\frametitle{Comparing Two Populations}
	
	\begin{itemize}[label={\color{blue}$\blacktriangleright$}]
		\item Testing $H_0 : p_1 - p_2 = D_0$ for $D_0 \neq 0$:
		\begin{itemize}[label={\color{blue}$\blacktriangleright$}]
			\item Calculate $Z$-statistic.
			\item Compare to $N(0,1)$ distribution.
			\item One or two-tailed depending on $H_1$.
		\end{itemize}
		
		\item Testing $H_0 : p_1 - p_2 = 0$:
		\begin{itemize}[label={\color{blue}$\blacktriangleright$}]
			\item Calculate combined proportion $\hat{p}$.
			\item Calculate $Z$-statistic.
			\item Compare to $N(0,1)$ distribution.
			\item One or two-tailed depending on $H_1$.
		\end{itemize}
	\end{itemize}
	
\end{frame}
\begin{frame}
	\frametitle{ANOVA}
	
	\begin{itemize}[label={\color{blue}$\blacktriangleright$}]
		\item Calculate sums of squares, degrees of freedom and mean squares for each source of variation.
		
		\item Calculate $F$-statistic.
		
		\item Compare to $F$-distribution with numerator degrees of freedom equal to the factor or interaction degrees of freedom, and denominator degrees of freedom equal to the error degrees of freedom.
		
		\item One-tailed, reject when $F$-statistic is too large.
	\end{itemize}
	
\end{frame}
\begin{frame}
	\frametitle{Simple Linear Regression}
	
	\begin{itemize}[label={\color{blue}$\blacktriangleright$}]
		\item Testing overall significance of model:
		\begin{itemize}[label={\color{blue}$\blacktriangleright$}]
			\item That is, testing $H_0 : \beta_1 = 0$ vs $H_1 : \beta_1 \neq 0$.
			\item Calculate $T$-statistic.
			\item Compare to $t$-distribution with $n-2$ degrees of freedom.
			\item Two-tailed.
			\item Can also test overall significance of model by testing $H_0 : \rho = 0$ vs $H_1 : \rho \neq 0$.
		\end{itemize}
	\end{itemize}
	
\end{frame}
\begin{frame}
	\frametitle{Simple Linear Regression}
	
	\begin{itemize}[label={\color{blue}$\blacktriangleright$}]
		\item General tests for $\beta_0$ and $\beta_1$:
		\begin{itemize}[label={\color{blue}$\blacktriangleright$}]
			\item Calculate $T$-statistic.
			\item Compare to $t$-distribution with $n-2$ degrees of freedom.
			\item One or two-tailed depending on $H_1$.
			\item May not be able to use $p$-value given in computer output, since this is the $p$-value for testing $H_0 : \beta_j = 0$ vs $H_1 : \beta_j \neq 0$, for $j = 0,1$.
		\end{itemize}
	\end{itemize}
	
\end{frame}
\begin{frame}
	\frametitle{Multiple Linear Regression}
	
	\begin{itemize}[label={\color{blue}$\blacktriangleright$}]
		\item Testing overall significance of model:
		\begin{itemize}[label={\color{blue}$\blacktriangleright$}]
			\item Calculate sums of squares, degrees of freedom and mean squares.
			\item Calculate $F$-statistic.
			\item Compare to $F$-distribution with $k$ numerator degrees of freedom and $n-k-1$ denominator degrees of freedom, where $k$ is the number of independent variables in the model.
			\item One-tailed, reject when $F$-statistic is too large.
		\end{itemize}
	\end{itemize}
	
\end{frame}
\begin{frame}
	\frametitle{Multiple Linear Regression}
	
	\begin{itemize}[label={\color{blue}$\blacktriangleright$}]
		\item Testing individual coefficient parameters:
		\begin{itemize}[label={\color{blue}$\blacktriangleright$}]
			\item Calculate $T$-statistic.
			\item Compare to $t$-distribution with $n-k-1$ degrees of freedom.
			\item One or two-tailed depending on $H_1$.
			\item Note that the $p$-value given in the computer output is for testing $H_0 : \beta_j = 0$ vs $H_1 : \beta_j \neq 0$, for $j = 0,\ldots,k$.
			\item The conclusion of each individual test is conditional on the fact that the other independent variables have already been included in the model.
		\end{itemize}
	\end{itemize}
	
\end{frame}
\begin{frame}
	\frametitle{Using Probability Tables}
	
	\begin{itemize}[label={\color{blue}$\blacktriangleright$}]
		\item Binomial tables:
		\begin{itemize}[label={\color{blue}$\blacktriangleright$}]
			\item Can only use the binomial tables if the required values of $n$ and $p$ are listed in the table.
			\item For values of $n$ and $p$ which are not in the table, you must use the binomial formula.
		\end{itemize}
		
		\item Normal tables:
		\begin{itemize}[label={\color{blue}$\blacktriangleright$}]
			\item When looking up a $Z$-value to find the corresponding probability, round the $Z$-value to 2 decimal places.
			\item When looking up a probability to find the corresponding $Z$-value, choose the closest probability.
			\item When probability falls exactly in the middle between two $Z$-values, choose mid-point (e.g., $z_{0.05} = 1.645$).
		\end{itemize}
	\end{itemize}
	
\end{frame}
\begin{frame}
	\frametitle{Using Probability Tables}
	
	\begin{itemize}[label={\color{blue}$\blacktriangleright$}]
		\item $t$-table:
		\begin{itemize}[label={\color{blue}$\blacktriangleright$}]
			\item To find a lower critical value, put a negative on the corresponding upper critical value.
		\end{itemize}
		
		\item $F$-tables:
		\begin{itemize}[label={\color{blue}$\blacktriangleright$}]
			\item Remember to use the correct $F$-table corresponding to the value of $\alpha$.
		\end{itemize}
		
		\item $t$-table and $F$-tables:
		\begin{itemize}[label={\color{blue}$\blacktriangleright$}]
			\item If you can't find the exact degree of freedom in the table, chose the closest degree of freedom.
		\end{itemize}
	\end{itemize}
	
\end{frame}
\end{document}