\documentclass[14pt]{beamer}
\usetheme{Boadilla}
\usepackage{booktabs}
\usepackage{enumitem}
\usepackage{tikz}

\newcommand{\E}{\mathbb{E}}
\usefonttheme{professionalfonts}
\usepackage{pgfplots}
\renewcommand{\arraystretch}{1.25}
\usetikzlibrary{trees}
\title[ECON2843]{Lecture 11}
\subtitle{Part 2 Probability and Distributions}
\date{}
\usepackage{amsmath,amssymb,mathtools,wasysym}
\begin{document}
	\begin{frame}
		\titlepage
	\end{frame}
	\begin{frame}
		\vspace{1cm}
		\centering
		{\color{blue}\large Sampling Distribution Part 2}
	\end{frame}
\begin{frame}
	\frametitle{Example 2}
	
	\begin{itemize}[label={\color{blue}$\blacktriangleright$}]
		\item Given a normal distribution with unknown mean $\mu$ and variance equal to 100, find the probability that the sample mean of a sample of 25 observations will differ from the population mean by less than 4 units.
		\item Let $\bar{X}$ denote the sample mean of the 25 observations.
		\item From the CLT we know that
		\[
		\bar{X} \sim N \left(\mu, \frac{\sigma^2}{n} = \frac{100}{25}\right)
		\]
	\end{itemize}
	
\end{frame}

\begin{frame}
	\frametitle{Solution}
	
	\begin{itemize}[label={\color{blue}$\blacktriangleright$}]
		\item We only care that the distance between the sample mean and population mean is less than 4 units.
		\item So we want to find $P(|\bar{X} - \mu| < 4)$:
	\end{itemize}
	
	\begin{align*}
		P(|\bar{X} - \mu| < 4) &= P(-4 < \bar{X} - \mu < 4) \\[1ex]
		&= P\left(\frac{-4}{\frac{10}{\sqrt{25}}} < \frac{\bar{X} - \mu}{\frac{\sigma}{\sqrt{n}}} < \frac{4}{\frac{10}{\sqrt{25}}}\right) \\[1ex]
		&= P(-2 < Z < 2) \\[1ex]
		&= 0.9544
	\end{align*}
	
\end{frame}

\begin{frame}
	\frametitle{Example 3}
	
	\begin{itemize}[label={\color{blue}$\blacktriangleright$}]
		\item The number of accidents per week at a hazardous intersection varies randomly with mean 2.2 and standard deviation 1.4.
		\item This distribution is discrete and certainly not normal.
		\item What is the approximate probability that there are fewer than 100 accidents at the intersection in a year?
	\end{itemize}
	
\end{frame}

\begin{frame}
	\frametitle{Solution}
	
	\begin{itemize}[label={\color{blue}$\blacktriangleright$}]
		\item Let $X_i$ be the number of accidents that occur in the $i$th week of the year.
		\item We know $E(X_i) = 2.2$ and $V(X_i) = 1.4^2$ for all $i$.
		\item We want to find the probability of fewer than 100 accidents in the year, i.e.:
	\end{itemize}
	
	\vspace{0.5em}
	
	\begin{align*}
		P\left(\sum_{i=1}^{52} X_i < 100\right) &= P\left(\frac{\sum_{i=1}^{52} X_i}{52} < \frac{100}{52}\right) \\[1ex]
		&= P(\bar{X} < 1.923)
	\end{align*}
	
\end{frame}

\begin{frame}
	\frametitle{Solution}
	
	\begin{itemize}[label={\color{blue}$\blacktriangleright$}]
		\item But we know from the CLT that
	\end{itemize}
	
	\[
	\bar{X} \sim N \left(\mu = 2.2, \frac{\sigma^2}{n} = \frac{1.4^2}{52}\right)
	\]
	
	\begin{itemize}[label={\color{blue}$\blacktriangleright$}]
		\item Therefore,
	\end{itemize}
	
	\begin{align*}
		P(\bar{X} < 1.923) &= P\left(\frac{\bar{X} - \mu}{\frac{\sigma}{\sqrt{n}}} < \frac{1.923 - 2.2}{\frac{1.4}{\sqrt{52}}}\right) \\[1ex]
		&= P(Z < -1.43) \\[1ex]
		&= 0.0764
	\end{align*}
	
\end{frame}

\begin{frame}
	\frametitle{Binomial Distribution}
	
	\begin{itemize}[label={\color{blue}$\blacktriangleright$}]
		\item Recall the binomial distribution that was introduced earlier.
		\item If $X\sim Bin(n,p)$, then $X$ was counting the number of successes in $n$ independent Bernoulli trials.
		\item We assumed that $p$ was known.
	\end{itemize}
	
\end{frame}
\begin{frame}
	\frametitle{Sample Proportion}
	
	\begin{itemize}[label={\color{blue}$\blacktriangleright$}]
\item But in reality, $p$ could be an unknown population parameter.
\item Therefore, just like we did with the population mean, we need to use a sample to estimate the population proportion $p$.
\item For example, suppose we are interested in whether Coke or Pepsi is the more popular soft drink.
\item Let $p$ denote the population proportion of people who prefer Coke over Pepsi.
	\end{itemize}
\end{frame}

\begin{frame}
	\frametitle{Sample Proportion}
	
	\begin{itemize}[label={\color{blue}$\blacktriangleright$}]
		\item If $X$ denotes the number of people who prefer Coke over Pepsi in a randomly selected sample, what is a reasonable estimate of $p$?
		\item If $n$ is the size of the sample, then a reasonable estimate of $p$ is the {\bf sample proportion} of people who prefer Coke over Pepsi:
		$$\hat{p}=\frac{X}{n}$$
		\item Let's investigate $\hat{p}$ in a little more detail...
	\end{itemize}
\end{frame}
\begin{frame}
	\frametitle{Sample Proportion}
	
	\begin{itemize}[label={\color{blue}$\blacktriangleright$}]
		\item Recall that we can also write $X = \sum_{i=1}^n X_i$, where
		
		\[
		X_i = \begin{cases}
			1 & \text{person $i$ prefers Coke over Pepsi} \\
			0 & \text{otherwise}
		\end{cases}
		\]
		
		\item So $\hat{p}$ is just the sample mean of a sample of independent Bernoulli random variables:
		
		\[
		\hat{p} = \frac{X}{n} = \frac{\sum_{i=1}^n X_i}{n}
		\]
	\end{itemize}
	
\end{frame}

\begin{frame}
	\frametitle{Sample Proportion}
	
	\begin{itemize}[label={\color{blue}$\blacktriangleright$}]
		\item Which means we can apply the CLT to find the sampling distribution of $\hat{p}$.
		\item And from last lecture we know that the mean and variance of $\hat{p}$ are equal to:
	\end{itemize}
	
	\vspace{0.5em}
	
	\[
	\mu_{\hat{p}} = E(\hat{p}) = \mu
	\]
	
	\[
	\sigma_{\hat{p}}^2 = V(\hat{p}) = \frac{\sigma^2}{n}
	\]
	
	\vspace{0.5em}
	
	where $\mu$ and $\sigma^2$ are the mean and variance, respectively, of the Bernoulli population $(X_i)$.
	
\end{frame}
\begin{frame}
	\frametitle{Probability Distribution of $X_i$}
	
	\begin{center}
		\begin{tabular}{ccc}
			\toprule
			$x_i$ & 0 & 1 \\
			\midrule
			$p(x_i)$ & $1-p$ & $p$ \\
			\bottomrule
		\end{tabular}
	\end{center}
	
	\vspace{1cm}
	
	\begin{columns}[T]
		\begin{column}{0.5\textwidth}
			$\mu = E(X_i)$ \\
			$\quad = 0 \times (1-p) + 1 \times p$ \\
			$\quad = p$
		\end{column}
		\begin{column}{0.5\textwidth}
			$\sigma^2 = V(X_i)$ \\
			$\quad = E(X_i^2) - (E(X_i))^2$ \\
			$\quad = (0^2 \times (1-p) + 1^2 \times p) - p^2$ \\
			$\quad = p(1-p)$
		\end{column}
	\end{columns}
	
\end{frame}

\begin{frame}
	\frametitle{Central Limit Theorem}
	
	\begin{itemize}[label={\color{blue}$\blacktriangleright$}]
		\item The CLT tells us that for sufficiently large $n$ (i.e., when both $np$ and $n(1-p)$ are $\geq 5$):
		
		\[
		\hat{p} \sim N \left(\mu_{\hat{p}} = p, \sigma_{\hat{p}}^2 = \frac{p(1-p)}{n}\right)
		\]
		
		\item If we standardise $\hat{p}$, we then get:
		
		\[
		\frac{\hat{p} - p}{\sqrt{\frac{p(1-p)}{n}}} = Z \sim N(0, 1)
		\]
	\end{itemize}
	
\end{frame}

\begin{frame}
	\frametitle{Example 1}
	
A psychologist believes that 80\% of male drivers when lost continue to drive hoping to find the location they seek rather than ask directions. To examine this belief, he took a random sample of 350 male drivers and asked each what they did when lost.

{\bf If the belief is true}, determine the probability that less than 75\% said they continue driving.
	
\end{frame}

\begin{frame}
	\frametitle{Solution}
	
	\begin{itemize}[label={\color{blue}$\blacktriangleright$}]
		\item Let $\hat{p}$ be the sample proportion of drivers that keep driving.
		\item We know that $n = 350$ and the population proportion is $p = 0.8$.
		\item We want to find $P(\hat{p} < 0.75)$.
		\item Since $n = 350$ is very large, we can apply the CLT and conclude that:
	\end{itemize}
	
	\vspace{0.5em}
	
	\[
	\hat{p} \sim N \left(p = 0.8, \frac{p(1-p)}{n} = \frac{0.8(1-0.8)}{350}\right)
	\]
	
\end{frame}

\begin{frame}
	\frametitle{Solution}
	
	\begin{itemize}[label={\color{blue}$\blacktriangleright$}]
		\item Therefore:
	\end{itemize}
	
	
	\begin{align*}
		P(\hat{p} < 0.75) &= P\left(\frac{\hat{p} - p}{\sqrt{\frac{p(1-p)}{n}}} < \frac{0.75 - 0.8}{\sqrt{\frac{0.8(1-0.8)}{350}}}\right) \\[1ex]
		&= P(Z < -2.34) \\[1ex]
		&= 0.0096
	\end{align*}
	
\end{frame}
\begin{frame}
	\frametitle{Example 2}
	
An accounting professor claims that no more than one-quarter of undergraduate business students will major in accounting. What is the probability that in a random sample of 1200 undergraduate business students, 336 or more will major in accounting?

\end{frame}

\begin{frame}
	\frametitle{Solution}
	
	\begin{itemize}[label={\color{blue}$\blacktriangleright$}]
		\item Let $X$ be the number of students that major in accounting out of the random sample of 1200.
		\item Then $X \sim Bin(n = 1200, p = 0.25)$.
		\item We want:
	\end{itemize}

	
	\begin{align*}
		P(X \geq 336) &= P(X = 336) + P(X = 337) + \ldots \\
		&\quad \ldots + P(X = 1200)
	\end{align*}
	
\end{frame}

\begin{frame}
	\frametitle{Solution}
	
	\begin{itemize}[label={\color{blue}$\blacktriangleright$}]
		\item We can use the CLT to approximate this probability:
	\end{itemize}
	
	\begin{align*}
		P(X \geq 336) &= P\left(\frac{X}{1200} \geq \frac{336}{1200}\right) \\ 
		&\approx P(\hat{p} > 0.28) \\ 
		&= P\left(\frac{\hat{p} - p}{\sqrt{\frac{p(1-p)}{n}}} > \frac{0.28 - 0.25}{\sqrt{\frac{0.25(1-0.25)}{1200}}}\right) \\ 
		&= P(Z > 2.40) \\ 
		&= 1 - 0.9918 \\ 
		&= 0.0082
	\end{align*}
	
\end{frame}


\end{document}